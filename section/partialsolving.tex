In this chapter, we will introduce the two solving approach that we explored. Solving can be performed using already computed-paths or a subgraph defined by paths.

Solving in both cases is basically performed using the same encoding defined in listing~\ref{lst:solver}.

\begin{minipage}[H]{\linewidth}
\begin{lstlisting}[style=mystyle, caption={Encoding of final solver}, label={lst:solver}, numbers=left, ,escapechar=|]
    time(1..horizon).

    at(R,P,0) :- start(R,P).

    { move(R,U,V,T) : nedge(U,V)} 1 :- agent(R), time(T).

    at(R,V,T) :- move(R,_,V,T).
            :- move(R,U,_,T), not at(R,U,T-1).

    at(R,V,T) :- 
        at(R,V,T-1), 
        not move(R,V,_,T), 
        time(T).

    :- {at(R,V,T)}!=1, agent(R), time(T).|\label{line:one_position_at_a_time}|

    :- { at(R,V,T) : agent(R) }  > 1, nvertex(V), time(T).
    :- move(_,U,V,T), move(_,V,U,T), U < V.

    goal_reached(R) :- at(R,V,horizon), goal(R,V). |\label{line:ipf_goal_reached}|

    #maximize{1,R : goal_reached(R)}. |\label{line:maximize_goal_reached}|
\end{lstlisting}
\end{minipage}


This encoding is derived from the MAPF encoding illustrated in listing~\ref{lst:base_mapf_encoding} with two significant modifications. The first modification involves the redefinition of movement predicates. Instead of relying on \(vertex/1\) and \(edges/2\), movement is now defined on \(nvertex/1\) and \(nedge/2\), which are generated from the encoding detailed in listing~\ref{lst:converting_paths_to_subgraph}.

The second distinction is highlighted in lines~\ref{line:ipf_goal_reached} and~\ref{line:maximize_goal_reached}. Unlike classical MAPF, which either provides a complete solution or returns unsatisfiable if no solution exists, the solver in this context can produce a partial solution. This means that due to the steps taken to reduce the complexity MAPF problem, there is a possibility that the solver might not find a complete solution.


In order to highlight the differences between the two different kinds of solving approaches we will present, we introduce an example denoted in the following figure~\ref{fig:partial_solving_example}. In this example, we have on the left a \(\tau\) result of IPF of a MAPF problem \(\mathcal{P}\).

\begin{figure}[H]
    \centering
    \caption{Example for solving approaches. Having a \(|\tau| = 4\) as result of IPF}\label{fig:partial_solving_example}
    \includegraphics[width=9cm]{img/partial_solving_example.png}
\end{figure}




\section{Pre-computed paths}
As mentioned in the Path Selection section~\ref{sec:pathselection}, the output can vary based on the specified objective. In the context of the primary objective, which aims to construct a (partial) plan, we utilize the set of \textbf{selected paths} to generate \textbf{pre-computed paths}. This translation is achieved through the following rules:

\begin{minipage}[H]{\linewidth}
\begin{lstlisting}[style=mystyle]
    at(R,V,T) :- 
        selected_path(R,I), 
        at(R,I,V,T).
\end{lstlisting}
\end{minipage} 
\noindent Through line~\ref{line:one_position_at_a_time}, we ensure that the selected paths are incorporated into the solution. Conversely, for agents without a selected path, the encoding is employed to compute their paths as the MAPF encoding described here~\ref{lst:base_mapf_encoding}.

\begin{figure}[H]
    \centering
    \caption{Possible output for Path Selection \& Pre computed paths}\label{fig:pre_computed_path_solving}
    \includegraphics[width=9cm]{img/pre_computed_path_solving.drawio.png}
\end{figure}

Figure~\ref{fig:pre_computed_path_solving} describes a partial plan \(\hat{\Pi}\) where only blue agent has a path. Using pre-computed paths aims to reduce the number of paths to compute. In the example outlined in figure~\ref{fig:pre_computed_path_solving}, the solving requires a makespan of five.

\section{Subgraph}

On the other hand, the second objective described in Section~\ref{sec:pathselection} aims to create a subgraph in order to reduce the size of the problem. As illustrated in Figure~\ref{fig:partial_solving_example}, we obtain the subgraph shown in Figure~\ref{fig:subgraph_solving}.

\begin{figure}[H]
    \centering
    \caption{Example for solving approaches. Having a \(|\tau| = 4\) as result of IPF}\label{fig:subgraph_solving}
    \includegraphics[width=9cm]{img/subgraph_solving.png}
\end{figure}

Contrary to pre-computing path, we aim to reduce the size of the graph the agents can move on. Applying MAPF on the problems described in figure~\ref{fig:subgraph_solving} can find a solution with a makespan of 4. To summarize both approaches, pre-computing paths aims to reduce the number of agent for which a path has to be found at the cost of a possible lost of optimality. On the other hand, using subgraphs reduce the space search of the initial problem \(\mathcal{P}\) by denoting a smaller problem \(\mathcal{P}'\). Note that if an optimal solution exists for \(\mathcal{P}'\) there is no guarantee that this solution is optimal for \(\mathcal{P}\).

The two approaches presented can be used as one. 

\subsection{Subgraphs Extension Strategies}

In practice, the subgraph solving approach seems to not be enough to fully solve instances; the agents seem to require too many additional time steps in order to ``solve conflict''. Thus, we introduce two strategies in order to extend subgraphs. 

\subsubsection{Corridor}

The corridor strategy is engineered to augment the subgraph-solving process by incorporating neighboring vertices and their associated edges directly into the sub-graph. Corridors can vary in size, allowing for different levels of expansion. Figure~\ref{img:corridor} illustrates a corridor with a size \(k\) equal to one. 


\begin{figure}[H]
    \centering
    \caption{Example of corridor}\label{img:corridor}
    \includegraphics[width=\widthimg]{img/corridor.drawio.png}
\end{figure}

The following encoding in listing~\ref{lst:corridor_encoding} describe how corridors for paths can be computed.

\begin{minipage}[H]{\linewidth}
\begin{lstlisting}[style=mystyle, caption={Corridor extension encoding}, label={lst:corridor_encoding}]
    #const corridor_level = 2.

    corridor(V,0) :- selected_path_for_corridor(R,I), at(R,I,V,_).
    
    corridor(V,K+1) :- 
        K < corridor_level,
        corridor(U,K),
        edge(U,V).

    nvertex(V) :- corridor(V,_).
\end{lstlisting}
\end{minipage}

Predicate \(selected\_path\_for\_corridor/2\) highlights a path that requires a corridor extension. In practice, we apply corridors to the conflicting paths added to the conflict-free set of paths.

With a sufficiently large \(k\), the entire graph can be covered; using a large corridor could essentially revert the problem back to a classical MAPF scenario.

In practice, corridors are created only for paths involved in conflicts, and a choice can be made to create corridors for only one of the two agents involved in a conflict.

\subsubsection{Diamond}

The diamond extension strategy involves expanding the sub-graph by incorporating diamond-shaped arrangements of vertices around initial vertices. As illustrated in figure~\ref{img:diamond}, various levels of diamond extension can be applied, each increasing the size of the sub-graph and thereby expanding the possibilities for conflict resolution.

\begin{figure}[H]
  \centering
  \caption{Example of diamond of size 1 and 2}\label{img:diamond}
  \includegraphics[width=\widthimg]{img/diamond.drawio.png}
\end{figure}


The following listing~\ref{lst:corridor_encoding} describes how diamonds are computed. As for corridors, the predicate \(selected\_vertex\_for\_diamond/1\) highlight a vertex that requires a diamond extension. In practice, we apply diamond extension on every conflict induced by the set of paths composing the subgraph. 

\begin{minipage}[H]{\linewidth}
\begin{lstlisting}[style=mystyle, caption={Diamond extension encoding}, label={lst:diamond_encoding}]
    #const diamond_level = 2.
    diamond(V,0) :- selected_vertex_for_diamond(V).
    
    diamond(V,S+1) :- 
        diamond(U,S), 
        S<diamond_level, 
        edge(U,V).
    
    nvertex(V) :- diamond(V,_).
\end{lstlisting}
\end{minipage}

